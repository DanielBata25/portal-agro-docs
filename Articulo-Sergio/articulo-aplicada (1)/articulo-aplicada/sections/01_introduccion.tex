\section{Introducción}

Manejar la información en el campo ha dejado de ser un simple valor agregado; hoy es un requisito de supervivencia y competitividad. En el Huila, existe una brecha histórica: tenemos una producción agrícola de altísima calidad, pero nos falta visibilidad en el mundo digital. Como bien sostienen \cite{garcia2007informacion}, la sostenibilidad de cualquier organización compleja depende de qué tan bien gestione su conocimiento. Fue esa necesidad la que impulsó la creación del Portal Agro-Comercial, una solución tecnológica pensada para conectar a las fincas —iniciando en Teruel— directamente con el mercado, sin intermediarios innecesarios.

El portal se construyó siguiendo el “manual”: un ciclo de vida de software estándar que cumplió con mostrar la oferta y gestionar usuarios. Pero hay una realidad: cuando las plataformas web crecen, los métodos tradicionales de despliegue se quedan cortos. La necesidad de lanzar mejoras sin que el servicio se caiga o se interrumpa se vuelve urgente. La industria lo sabe, y por eso el despliegue continuo que mencionan Meyer y Schmitt \cite{meyer2016continuous} es fundamental para perder el miedo a romper el sistema con cada actualización.

Aquí es donde las \textit{Feature Toggles} (o banderas de funcionalidad) juegan un papel crucial. Más que una herramienta, son una estrategia arquitectónica que permite cambiar el comportamiento del software sin tocar el código base ni hacer despliegues completos \cite{rahman2016msr}. Aunque nuestro Portal Agro-Comercial no nació con esta arquitectura, al analizar su estructura actual vemos que es el candidato perfecto para hacer esta transición.

La literatura reciente nos da la razón. Estudios como los de \cite{garcia2021practitioners} muestran que los toggles no son solo técnica; cambian la cultura del equipo al permitirnos separar el momento en que instalamos el código (despliegue) del momento en que el usuario lo usa (liberación).

Este artículo busca dos cosas puntuales. Primero, documentar cómo está construido hoy el Portal Agro-Comercial como un caso real de digitalización rural. Y segundo, usar este proyecto como base para proponer un modelo de implementación de \textit{Feature Toggles}, respaldado por una revisión de 20 artículos especializados. Queremos demostrar que integrar esta técnica es la respuesta a la deuda técnica y la complejidad futura que advierten \cite{sharma2024complexity}, asegurando así que la plataforma pueda evolucionar sin colapsar.
