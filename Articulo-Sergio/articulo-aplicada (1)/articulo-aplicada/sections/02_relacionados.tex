\section{Trabajos Relacionados}

La evolución del Portal Agro-Comercial no ocurre en el vacío; responde a un entorno de ingeniería de software donde la velocidad de entrega es lo que marca la diferencia. Para darle una base sólida a nuestra propuesta técnica, no nos limitamos a revisar la teoría, sino que analizamos la literatura reciente bajo tres lentes: cómo se despliega hoy en día, qué tan complejo es mantenerlo y el gran desafío de la deuda técnica al usar \textit{Feature Toggles}.

\subsection{Estrategias de Despliegue y Configuración Dinámica}

Pasar de webs monolíticas a arquitecturas vivas y dinámicas es una tendencia que ya no tiene vuelta atrás. \cite{meyer2016continuous} son claros al respecto: la entrega continua (\textit{Continuous Delivery}) ya no es un lujo opcional, sino un requisito para que los sistemas modernos no colapsen. Siguiendo esa lógica, \cite{ramaswamy2024zerodowntime} argumentan que si queremos ese anhelado ``tiempo de inactividad cero'', es obligatorio separar el acto de instalar código del acto de activarlo. Y ahí es donde las \textit{Feature Flags} se convierten en la pieza clave.

Pero esto no es solo un tema del servidor (backend). \cite{lee2025frontend} ponen el foco en el frontend, algo vital para nuestro Portal Agro-Comercial. Imaginemos poder actualizar la vista de los productos sin que al agricultor se le recargue toda la página; esa es la promesa. Además, \cite{esther2025microservices} refuerzan la idea de que, en entornos de microservicios, estas configuraciones dinámicas son lo que mantiene la resiliencia del sistema.

\subsection{Complejidad y Gestión de Deuda Técnica}

Implementar \textit{Feature Toggles} tiene su costo. \cite{rahman2016msr}, en uno de los estudios más citados, advierten que llenar el código de toggles sin control lo vuelve inmanejable. Esta preocupación sigue vigente: \cite{sharma2024complexity} encontraron recientemente una relación directa entre la cantidad de toggles activos y la complejidad ciclomática, generando mayores dificultades de mantenimiento.

De ahí surge el concepto de ``Deuda de Toggles'', abordado por \cite{ortiz2021toggledebt}, quien insiste en que debemos saber cuándo una bandera ya cumplió su propósito y debe eliminarse. Para el Portal, esto es crítico: la limpieza. \cite{sridharan2020piranha} muestran el ejemplo de Uber con su herramienta Piranha, probando que automatizar esta limpieza es posible. El problema, según \cite{abdalkareem2021removal}, es humano: los desarrolladores suelen posponer la eliminación de toggles viejos por miedo a romper algo, y así es como el riesgo se acumula silenciosamente.

\subsection{Prácticas en la Industria y Taxonomías}

Para implementar correctamente toggles en el Portal, es necesario observar prácticas industriales consolidadas. \cite{rahman2019chrome} analizaron Google Chrome y evidenciaron que su modularidad depende profundamente de un sistema masivo de toggles. En un contexto similar, \cite{smith2022interdependencies} estudiaron Microsoft Office y resaltaron un riesgo relevante: las interdependencias. Activar un toggle puede requerir que otro también esté habilitado, creando una red que \cite{hal2022interaction} denominan ``interacción de features''.

Finalmente, clasificar correctamente los toggles resulta esencial. \cite{chen2025horizon} proponen el marco HORIZON para categorizar tipos de toggles, mientras que \cite{ajmeri2022heuristics} sugieren tratarlos como ``Toggles como Código''. La enseñanza clave es que no todos los toggles son iguales: no es lo mismo un permiso de administrador que un experimento temporal, y esta distinción fundamentará nuestra arquitectura propuesta.

\begin{table}[htbp]
\centering
\caption{Clasificación y Propósito de Feature Toggles según la Literatura}
\label{tab:toggle_types}
\small
\begin{tabular}{p{2.5cm} p{3.5cm} p{1.5cm}}
\toprule
\textbf{Tipo de Toggle} & \textbf{Propósito Principal} & \textbf{Duración} \\
\midrule
Release Toggles & Desacoplar despliegue de liberación de funciones \cite{rahman2016msr}. & Corta \\
Business Toggles & Habilitar funciones premium o por segmento de usuario \cite{chen2025horizon}. & Larga \\
Ops Toggles & Controlar aspectos operativos bajo carga (ej. deshabilitar reportes pesados) \cite{ramaswamy2024zerodowntime}. & Media \\
Permission Toggles & Modificar comportamiento según rol (Admin/Productor) \cite{ajmeri2022heuristics}. & Larga \\
\bottomrule
\end{tabular}
\end{table}