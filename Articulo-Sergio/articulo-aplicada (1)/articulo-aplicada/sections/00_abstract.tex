Llevar la transformaci\u00f3n digital al campo no es solo cuesti\u00f3n de crear herramientas accesibles, sino de garantizar que sean t\u00e9cnicamente sostenibles a largo plazo. Este art\u00edculo presenta el desarrollo del Portal Agro-Comercial del Huila, una plataforma web creada para dar visibilidad a la producci\u00f3n de las fincas del departamento y conectarlas directamente con el mercado.

Si bien este aplicativo se cre\u00f3 con los esquemas tradicionales, este trabajo plantea un cambio de paradigma hacia el uso de \textit{Feature Toggles} (banderas de funcionalidad). Bas\u00e1ndonos en la literatura reciente, proponemos una arquitectura evolutiva que integre esta t\u00e9cnica. El gran objetivo es separar el despliegue t\u00e9cnico de la liberaci\u00f3n de funciones; esto permitir\u00e1, en el futuro, realizar entregas continuas, minimizar riesgos en actualizaciones cr\u00edticas y gestionar la deuda t\u00e9cnica de forma eficiente, todo ello sin interrumpir el servicio vital para agricultores y compradores.
