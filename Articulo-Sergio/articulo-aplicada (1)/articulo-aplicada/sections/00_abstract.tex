Este artículo presenta la aplicación de una estrategia de despliegue basada en *feature toggles* en el proyecto formativo “Portal Agro-Comercial del Huila”, una plataforma digital desarrollada para conectar a productores agropecuarios y consumidores en Colombia. El principal desafío de ingeniería del sistema radica en su naturaleza evolutiva: inicia como un prototipo desplegado en un solo municipio y se expande progresivamente a toda la región. Para abordarlo, se analizó un marco teórico compuesto por veinte estudios académicos e industriales sobre *feature toggles*. Los resultados demuestran que estos pueden emplearse no solo como estructuras condicionales, sino como un mecanismo arquitectónico para controlar la activación de funcionalidades, habilitar la integración y entrega continua (CI/CD), los lanzamientos tipo *canary* y la segmentación por roles. La metodología propuesta se alinea con las mejores prácticas de DevOps, reduciendo el riesgo de despliegue y favoreciendo una evolución sostenible del software en contextos educativos y de innovación rural.
